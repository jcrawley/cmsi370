%
% latex-sample.tex
%
% This LaTeX source file provides a template for a typical research paper.
%

%
% Use the standard article template.
%
\documentclass{article}

% The geometry package allows for easy page formatting.
\usepackage{geometry}
\geometry{letterpaper}

% Load up special logo commands.
\usepackage{doc}

% Package for formatting URLs.
\usepackage{url}

% Packages and definitions for graphics files.
\usepackage{graphicx}
\usepackage{epstopdf}
\DeclareGraphicsRule{.tif}{png}{.png}{`convert #1 `dirname #1`/`basename #1 .tif`.png}

%
% Set the title, author, and date.
%
\title{Dating Website Interfaces}
\author{Joseph Crawley}
\date{September 18, 2012}

%
% The document proper.
%
\begin{document}

% Add the title section.
\maketitle

% Add an abstract.
\abstract{}
For this assignment, Terran Moore and I looked how Match.com, Eharmony.com, and ChristianMingle.com measured in the usability metrics of Learnability, Satisfaction, and Efficiency. For this assignment, we had 5 users create profiles and complete similar tasks in order to measure these metrics.  

% Add various lists on new pages.
\pagebreak
\tableofcontents


%\listoffigures


%\listoftables

% Start the paper on a new page.
\pagebreak

%
% Body text.
%
\section{Learnability}
\label{Learnability}

\subsection{E-harmony.com}
Users were able to fill out profiles almost instantly. Searching was hard to understand and users became frustrated with the searching of the system.

\subsection{Match.com}
Menus were easy to understand and users could fill out profiles almost instantly. The searching function was easy to pick up on for users, and users could navigate the site with ease.

\subsection{ChristianMingle.com}
The site was very simple and easy to understand. Menus were clearly marked, and everything was labeled very quickly.


\section{Efficiency}
\subsection{E-harmony.com}
E-harmony was efficient on every test except for search ability, which it tested very poorly in. Its search ability was foreign to users who were familiar with networking sites and caused users to be frustrated.

\subsection{Match.com}
Match.com was very intuitive to users for all aspects. Users felt that they could navigate the site quickly and efficiently with little errors and with a low learning curve.

\subsection{ChristianMingle.com}
Users liked how simple ChristianMingle.com was. They felt that the simplicity helped them navigate the site easily. The only thing that was not simple was updating profile information, which took users a few more clicks than necessary.


\section{Satisfaction}
\subsection{E-harmony.com}
Users felt this site was very good for something, but bad for dating itself. Users felt that it was next to impossible and very timely to connect with matches. Users had low satisfaction with the site.


\subsection{Match.com}
	Users felt that this was the preferred choice in dating websites. It combined the comprehensiveness of E-harmony with the comprehensiveness of Match.com. Users were happy with how quick they picked up the site and how natural the interface was.

\subsection{ChristianMingle.com}
Users felt Christian Mingle was easy to use, but too simple. Users did not feel that the creating a profile interface was comprehensive enough. Although the menus were simple, they felt that the actions they wanted to take were not always available in the menus.

\section{Conclusion}

	A command line interface can be intimidating at first, but can give the user a sense of comfortability. While typing in options rather than clicking them is unconventional, it is more efficient and pleasing for the user. While this style interface is difficult to design and code, the end result leaves the user with a mental model that can be easily understood by multiple users of different backgrounds and technology experience. While this interface may not be the most efficient for all websites and products, the command line is the ideal way to talk to Headmaster so information is accessed quickly.
	
\pagebreak
% Generate the bibliography.
\bibliography{intro}
\bibliographystyle{plain}

\end{document}
